\documentclass[]{paper}
\usepackage[utf8]{inputenc}
\usepackage{fixltx2e}

\begin{document}

\title{Manual do Usuário \\ Módulo de Física - RKE v0.1}
\author{Faca Vermelha Estúdios\\João da Silva, Marina Salles, Ricardo Macedo}
\date{Today}
\maketitle

\section{Introdução}
Este é o pequeno manual de usuário para o Módulo de Física do Red Knife Engine. Aproveite!
\section{Uso}
O Módulo de Física vem com um testador para facilitar o uso. Para utilizá-lo, basta chamá-lo em linha de comando:
\\

\texttt{./rke-tmf}

\subsection{Argumentos}
O testador possui alguns argumentos que devem ser colocados para que a simulação seja executada. São estes:
\begin{description}
\item[-f \emph{nome do arquivo}] Especifica o \textbf{arquivo} com as condições iniciais a ser carregado.
\item[-d \emph{valor}] Especifica a \textbf{granularidade} (o tamanho do quanta de tempo) da simulação. Este valor é dado em milissegundos. \emph{[Padrão = 1000]}
\item[-n \emph{valor}] Especifica o número de \textbf{iterações} que o módulo rodará. \emph{[Padrão = 1]}
\item[-a \emph{valor}] Especifica o coeficiente de \textbf{arrasto} da superfície. Este valor deve estar entre 0.0 e 1.0. \emph{[Padrão = 0.0]}
\end{description}

\section{Formato do arquivo}
O formato do arquivo, tanto de entrada quanto o de saída, seguem a seguinte especificação:

\subsection{.w}
Esta seção indica valores do vetor vento. Possui duas componentes:
\begin{description}
\item[vwx] Componente $x$ do vento
\item[vwy] Componente $y$ do vento
\end{description}
\subsubsection{Exemplo}
\texttt{.w\\
0.5\\
0.5}

\subsection{.b}
Esta é a seção que especifica as bombas presentes no mundo. Após o código \texttt{.b}, indica-se o número de bombas presentes na seção.

Cada linha representa uma bomba, e cada bomba possui os seguintes parâmetros:
\begin{description}
\item[k\textsubscript{n}] Identificador único da $n$-ésima bomba.
\item[t\textsubscript{n}] Tempo de vida (em segundos) da $n$-ésima bomba.
\item[x\textsubscript{n}] Componente $x$ da $n$-ésima bomba.
\item[y\textsubscript{n}] Componente $y$ da $n$-ésima bomba.
\end{description}
\subsubsection{Exemplo}
\texttt{.b 2\\
1 5 2 4\\
2 10 3 2}

\subsection{.s}
Esta seção especifica os parâmetros do navio. Possui cinco componentes:
\begin{description}
\item[m] Massa do navio.
\item[sx] Posição $x$ do navio.
\item[sy] Posição $y$ do navio.
\item[vsx] Componente $x$ da velocidade do navio.
\item[vsy] Componente $y$ da velocidade do navio.
\end{description}

\subsubsection{Exemplo}
\texttt{.s\\
5.2\\
2\\
2\\
1\\
3}

\subsection{Arquivo de exemplo}
\texttt{.w\\
0.5\\
0.5\\
.b 2\\
1 5 2 4\\
2 10 3 2\\
.s\\
5.2\\
2\\
2\\
1\\
3}


\section{Saída do testador}
O testador, dada as condições iniciais carregadas e os argumentos de linha de comando dados, grava um arquivo \texttt{saida.out} com o formato acima descrito contendo as informações do estado do mundo após as iterações da simulação.

\section{Manual do desenvolvedor}
Se você tem interesse em descobrir como funciona o Módulo de Física do Red Knife Engine, neste mesmo diretório, consulte os subdiretórios \texttt{/html} e \texttt{/latex} para a documentação em HTML e PDF, respectivamente.

\end{document}